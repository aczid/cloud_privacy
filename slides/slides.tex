\documentclass{beamer}

\usepackage[english]{babel}
\usepackage{graphicx,hyperref,ru,url}

\title{Privacy in Cloud Computing}

\author{Aram Verstegen \and Erik Boss}

\institute{Radboud University Nijmegen}

\date{\today}

\begin{document}

\begin{frame}
  \titlepage
\end{frame}

\begin{frame}
  \frametitle{Outline}
  \tableofcontents
\end{frame}

% These frametitles are kinda tongue-in-cheek, not sure if I want to keep them -- Erik

\section{Introduction}
% I estimate this section at about 20 minutes (depending on amound questions)

\begin{frame}
  \frametitle{Outline}
  \tableofcontents[currentsection,currentsubsection]
\end{frame}

\begin{frame}
    \frametitle{What are we dealing with?}
    %/* General definitions, explanations on cloud computing: SaaS/PaaS/IaaS
    %(with a picture?), public/private/community/hybrid clouds  */
    \begin{itemize}	
    \item \small Software as a Service (SaaS)
      \scriptsize{Webmail, online storage, social media}
    \item \small Platform as a Service (PaaS)
      \scriptsize{Virtual application platforms}
    \item \small Infrastructure as a Service (IaaS)
      \scriptsize{Scalable computing and storage services}
    \end{itemize}
    /* Network as a service */
\end{frame}

\begin{frame}
    \frametitle{Who are we dealing with?}
    Roles:
    \begin{itemize} % We might put this in a pyramid diagram
      \item Cloud service providers that offer ?aaS
      \item Cloud service users that build services on that
      \item Customers of these services
    \end{itemize}

    Cloud Service providers:
    \begin{itemize}	
    \item \small SaaS
      \scriptsize{Google, Apple, Microsoft}
    \item \small PaaS
      \scriptsize{Google AppEngine, Amazon AWS, Microsoft Azure (SQL), EngineYard, Heroku}
    \item \small IaaS
      \scriptsize{Amazon EC2 / S3, Google Compute Engine / BigQuery, Linode}
    \end{itemize}

    % Cloud service users: some sites you use, customers: you
\end{frame}

\begin{frame}
    \frametitle{Why does the industry like it?}
    %/* Why use cloud computing at all; virtualization, scaling, fast
    %deployment, pay-for-use, etc  */
    \begin{itemize}	
    \item Cost
    \item Fast wide scalability
    \item Pay as you go
    \item Outsource responsibility
    \item Buzz
    \end{itemize}
\end{frame}

\begin{frame}
    \frametitle{Why should we be skeptical?}
    %/* What are the problems; general overview of the security requirements and
    %problems (depending on type of cloud); picture(s) about cloud security for
    %fun and profit */
    \begin{itemize}	
      \item Centralized assets
      \item Less diversity
      \item Shared environment
      \item Logical segregation
      \item Usually has better security
      \item Major impact when breaches do occur
    \end{itemize}
\end{frame}

\begin{frame}
    \frametitle{Might your data be in peril?}
    %/* Global overview on specific privacy concerns w.r.t. clouds; example(s)
    %of how things can go badly; primarily: why does it matter! */
    \begin{itemize}	
      \item ``Real name policy''
      \item Data ownership
      \item Local legislation % For example: age of consent in various countries - flickr picture legal one place, illegal in another
      %\item Piracy claims vs Fair use % (YouTube knows about that song you ripped off and is obliged to tell the authorities)
      \item The declining cost of storing \textbf{everything}
      \item SLAs
      \item \dots
    \end{itemize}
\end{frame}

\begin{frame}
    \frametitle{Provider viewpoints}
    \begin{itemize}
    \item Google: ``Onze menukaart is beperkt, maar je kunt toch weg? Dan moet je verder niet zeiken'' (``Google weigert data in Europa te houden'' - Webwereld, 2011)
    \item Microsoft: ``Microsoft cannot provide those guarantees. Neither can any other company'' (``Microsoft admits Patriot Act can access EU-based cloud data'' - ZDnet, 2011)
    \item Amazon: ``Customers can choose to store all data in the EU by using the EU (Ireland) Amazon S3 Region.'' (Amazon S3 FAQ)
    \end{itemize}
\end{frame}

\begin{frame}
    \frametitle{Data locations}
    \begin{itemize}
    \item OV chipkaart: Talys, France
    \end{itemize}
\end{frame}

\section{Legal aspects}
% Another 20 minutes should be doable

\begin{frame}
  \frametitle{Outline}
  \tableofcontents[currentsection,currentsubsection]
\end{frame}

\begin{frame}
    \frametitle{Because a little reminder is always nice}
    %/* Quick recap of the most important and relevant legislation, i.e., data
    %protection and privacy directives. */
    \begin{itemize}	
    \item EU Data protection directive 95/46/EC
    \item Safe Harbor principle
    \item PATRIOT act
    \item \dots
    \end{itemize}
\end{frame}

\begin{frame}
    \frametitle{Safe harbor principle}
    \begin{scriptsize}
    \begin{itemize}
    \item \textbf{Notice} - Individuals must be informed that their data is being collected and about how it will be used.
    \item \textbf{Choice} - Individuals must have the ability to opt out of the collection and forward transfer of the data to third parties.
    \item \textbf{Onward Transfer} - Transfers of data to third parties may only occur to other organizations that follow adequate data protection principles.
    \item \textbf{Security} - Reasonable efforts must be made to prevent loss of collected information.
    \item \textbf{Data Integrity} - Data must be relevant and reliable for the purpose it was collected for.
    \item \textbf{Access} - Individuals must be able to access information held about them, and correct or delete it if it is inaccurate.
    \item \textbf{Enforcement} - There must be effective means of enforcing these rules.
    \end{itemize}
    \end{scriptsize}
    \footnotesize{Source: \url{{http://en.wikipedia.org/wiki/International_Safe_Harbor_Privacy_Principles}}}
\end{frame}

\begin{frame}
    \frametitle{As always: "it depends"}
    /* Legal analysis differs given the type of cloud computing (so break it
    down), it matters whether we have public or private clouds, IaaS or SaaS,
    etc. */
    % Erik, ik weet niet welke kant je hiermee op wilt - laat ik aan jou over
\end{frame}

\begin{frame}
    \frametitle{Service Level Agreements (SLAs)}
    \begin{itemize}
    \item Contract law
    \item May require you to:
      \begin{itemize} % http://www.networkworld.com/news/2012/120612-hp-amazon-cloud-264847.html?page=2
        \item occupy two mirrored instances at different datacenters
        \item may allow one of these to go down at points
        \item may give different guarantees for different aspects
      \end{itemize}
    \item ...
    \end{itemize}
\end{frame}

\begin{frame}
    \frametitle{There is more to law than code}
    /* If we do manage to find relevant case law, put it here */
\end{frame}

% Perhaps this is a good one to encite a discussion
% Sure, so I've moved the discussion slide after this one
\begin{frame}
    \frametitle{Strategic defense consideration}
    \begin{itemize}
      \item Large concentrations of valuable data
      \item hosted on shared infrastructure
      \item (with a broad and deep attack surface)
      \item on a platform that offers tremendous computing capacity
      \item to anybody with a credit card
    \end{itemize}
\end{frame}

\begin{frame}
    \frametitle{Reinventing the wheel}
    /* Some room for discussion: how does our audience feel about the legal
    bits and pieces? */
\end{frame}

\section{Technical aspects}

\begin{frame}
  \frametitle{Outline}
  \tableofcontents[currentsection,currentsubsection]
\end{frame}

\begin{frame}
    \frametitle{To trust or not to trust}
    %/* In-depth technical privacy in cloud setting; What does privacy mean in
    %this context, how can it be achieved (in general): all sorts of things are involved
    %(network security, web security, encryption, computing under encryption,
    %etc) */

    Availability: \checkmark \\
    Focus on:
    \begin{itemize}
      \item Confidentiality
      \item Integrity
    \end{itemize}
    We apply:
    \begin{itemize} % Note these are ordered from the most ephemeral to the most real
      \item Identity management
      \item Application security
      \item System \& Network security
      \item Physical security
      \item Hypervisor security % Isn't physical security arguably more 'real' than hypervisor security?
    \end{itemize}
\end{frame}

\begin{frame}
    \frametitle{Identity management}
    Prevent identity fraud:
    \begin{itemize}
      \item Single sign-on (SSO) % Maybe not this... but it's related
      \item Multi-factor authentication
      \item Password policies, security questions
      %\item SSL validity
    \end{itemize}
\end{frame}

\begin{frame}
    \frametitle{Application security}
    Prevent software abuse:
    \begin{itemize}
      \item Input validation
      \item Secure storage
      \item Avoid logical bugs
      \item Audit logging
      \item Application software patches
    \end{itemize}
\end{frame}

\begin{frame}
    \frametitle{System \& Network Security}
    Prevent network abuse:
    \begin{itemize}
      \item Firewalling
      \item Secure tunnels % Maybe skip this
      \item Airgapped management network
      \item Central audit logging
      \item Intrusion detection services (IDS)
      \item System software patches (fabric management)
    \end{itemize}
\end{frame}

\begin{frame}
    \frametitle{Physical Security}
    Prevent physical entry:
    \begin{itemize}
      \item Fences
      \item Locks
      \item Guards
      \item Cameras
      \item Badges \& Secure doors
      \item Electrically safe fire extinguishers (inert gas)
      \item Redundant power, backup power
      \item Redundant cooling water
    \end{itemize}
\end{frame}

\begin{frame}
    \frametitle{Hypervisor security}
    % This is an awesome paper about backdoored C compilers
    ``Reflections on Trusting Trust'' (Kevin Thompson, 1991) \\
    Modern equivalent: cloudburst exploit \\
    Prevent breaking out of virtual machines:
    \begin{itemize}
      \item Fabric management
      \item Host-based intrusion detection
      \item \dots
    \end{itemize}
\end{frame}

\begin{frame}
    \frametitle{The best defense is a good\dots}
    /* Attacks, attacks and more attacks; If possible, frighten the audience!
    Malicious images, simple(r) web application vulnerabilities, etc */
    \begin{itemize}
      \item Poisoning the well (Harron Meer, 2009)
      \item Cracking in the cloud (Various) % TODO cite
      \item Gigabit DDoS as a service % 250Mbit/s per node!
      \item Linode hack
    \end{itemize}
\end{frame}

\begin{frame}
    \frametitle{The best offense is a good\dots}
    /* Countermeasures; PETs; best practices; more detailed explanations of
    systems that give privacy guarantees */
    \begin{itemize}
      \item Only store encrypted data and let the client encrypt it on their machine
      \item Make use of homomorphic encryption to be able to work with encrypted data
      \item Don't log sensitive data
    \end{itemize}
\end{frame}

\begin{frame}
    \frametitle{Case study: private cyberlocker}
    \begin{itemize}
    \item Secure encrypted data
    \item Deduplication
    \item Key management
    \end{itemize}
\end{frame}

\begin{frame}
    \frametitle{Secure encrypted data}
    \begin{itemize}
    \item Open source design
    \item No use of driver blobs, browser plugins or server-side code
    \item Encryption/decryption only happens in the client
    \item Keys only available on client(s)
    \end{itemize}
\end{frame}

%\begin{frame}
    %\frametitle{Deduplication of encrypted data}
    %Use a checksum of the plaintext leaks information about what is stored.
%\end{frame}

\begin{frame}
    \frametitle{Key management}
    Make backup keys
    
    Private:
    $k_A = genKey()$
    $k_B = genKey()$
    \dots

    $o = hash(p)$

    Public:
    $o_A = E(o, k_A)$
    $o_B = E(o, k_B)$
    \dots

    $c = E(p,o)$
    % Each file has a different key
    % Low overhead
    % Flexible to add new keys/owners
\end{frame}

\begin{frame}
    \frametitle{What is stored}
    \begin{itemize}
    \item Encrypted blob with metadata, checksum
    \item Plaintext's checksum encrypted with user keys
    \end{itemize}
\end{frame}

\begin{frame}
    \frametitle{Results}
    \begin{itemize}
    \item Usable in public and private cloud
    \item Easy deduplication
    \item Add a new (device key) with your old key
    \item Poor backward security
    \end{itemize}
\end{frame}

\begin{frame}
    \frametitle{I can already see the ending}
    /* Obligatory 'Questions?' slide, possibly with a funny picture. */
\end{frame}

\end{document}
